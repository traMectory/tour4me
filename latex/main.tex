%%
%% This is file `sample-acmsmall-biblatex.tex',
%% generated with the docstrip utility.
%%
%% The original source files were:
%%
%% samples.dtx  (with options: `acmsmall-biblatex')
%% 
%% IMPORTANT NOTICE:
%% 
%% For the copyright see the source file.
%% 
%% Any modified versions of this file must be renamed
%% with new filenames distinct from sample-acmsmall-biblatex.tex.
%% 
%% For distribution of the original source see the terms
%% for copying and modification in the file samples.dtx.
%% 
%% This generated file may be distributed as long as the
%% original source files, as listed above, are part of the
%% same distribution. (The sources need not necessarily be
%% in the same archive or directory.)
%%
%%
%% Commands for TeXCount
%TC:macro \cite [option:text,text]
%TC:macro \citep [option:text,text]
%TC:macro \citet [option:text,text]
%TC:envir table 0 1
%TC:envir table* 0 1
%TC:envir tabular [ignore] word
%TC:envir displaymath 0 word
%TC:envir math 0 word
%TC:envir comment 0 0
%%
%%
%% The first command in your LaTeX source must be the \documentclass command.
\documentclass[sigconf,natbib=false]{acmart}


%%
%% \BibTeX command to typeset BibTeX logo in the docs
\AtBeginDocument{%
  \providecommand\BibTeX{{%
    Bib\TeX}}}

%% Rights management information.  This information is sent to you
%% when you complete the rights form.  These commands have SAMPLE
%% values in them; it is your responsibility as an author to replace
%% the commands and values with those provided to you when you
%% complete the rights form.
\setcopyright{acmcopyright}
\copyrightyear{2018}
\acmYear{2018}
\acmDOI{XXXXXXX.XXXXXXX}


%%
%% These commands are for a JOURNAL article.
\acmJournal{JACM}
\acmVolume{37}
\acmNumber{4}
\acmArticle{111}
\acmMonth{8}

%%
%% Submission ID.
%% Use this when submitting an article to a sponsored event. You'll
%% receive a unique submission ID from the organizers
%% of the event, and this ID should be used as the parameter to this command.
%%\acmSubmissionID{123-A56-BU3}

%%
%% For managing citations, it is recommended to use bibliography
%% files in BibTeX format.
%%
%% You can then either use BibTeX with the ACM-Reference-Format style,
%% or BibLaTeX with the acmnumeric or acmauthoryear sytles, that include
%% support for advanced citation of software artefact from the
%% biblatex-software package, also separately available on CTAN.
%%
%% Look at the sample-*-biblatex.tex files for templates showcasing
%% the biblatex styles.
%%


%%
%% The majority of ACM publications use numbered citations and
%% references, obtained by selecting the acmnumeric BibLaTeX style.
%% The acmauthoryear BibLaTeX style switches to the "author year" style.
%%
%% If you are preparing content for an event
%% sponsored by ACM SIGGRAPH, you must use the acmauthoryear style of
%% citations and references.
%%
%% Bibliography style
\RequirePackage[
  datamodel=acmdatamodel,
  style=acmauthoryear,
  ]{biblatex}

\usepackage{xspace}

%% Declare bibliography sources (one \addbibresource command per source)
\addbibresource{software.bib}
\addbibresource{sample-base.bib}


%%
%% end of the preamble, start of the body of the document source.
\begin{document}

%%
%% The "title" command has an optional parameter,
%% allowing the author to define a "short title" to be used in page headers.
\title{Tour4Me: A Framework for Customized Tour Planning Algorithms}

%%
%% The "author" command and its associated commands are used to define
%% the authors and their affiliations.
%% Of note is the shared affiliation of the first two authors, and the
%% "authornote" and "authornotemark" commands
%% used to denote shared contribution to the research.

\author{Kevin Buchin}
\orcid{1234-5678-9012}
\affiliation{%
  \institution{TU Dortmund}
  \city{Dortmund}
  \country{Germany}
}
\email{kevin.buchin@tu-dortmund.de}

\author{Mart Hagedoorn}
\orcid{1234-5678-9012}
\affiliation{%
  \institution{TU Dortmund}
  \city{Dortmund}
  \country{Germany}
}
\email{mart.hagedoorn@tu-dortmund.de}

\author{Guangping Li}
\orcid{1234-5678-9012}
\affiliation{%
  \institution{TU Dortmund}
  \city{Dortmund}
  \country{Germany}
}
\email{guangping.li@tu-dortmund.de}

%%
%% By default, the full list of authors will be used in the page
%% headers. Often, this list is too long, and will overlap
%% other information printed in the page headers. This command allows
%% the author to define a more concise list
%% of authors' names for this purpose.
\renewcommand{\shortauthors}{Buchin, Hagedoorn, and Li}
\newcommand{\tM}{\textsc{Tour4Me}\xspace}

%%
%% The abstract is a short summary of the work to be presented in the
%% article.
\begin{abstract}
The touring problem aims to find an `interesting' (round) trip of a given length. Here, what is considered interesting depends on the type of the desired route, e.g., a user may be looking for a off-road cycling trip or fast running route.
There are two main perspectives on the touring problem, maximizing profit or minimizing cost, which result in very different algorithmic solutions. We provide a framework that allows for straightforward integration of new algorithms for both perspectives on the touring problem.
In this demonstration we have included a new exact solver, a heuristic, and two greedy methods. The user can experiment with the algorithms and different profits/costs. The generated tours can be explored in an easy-to-use web interface.
\end{abstract}

%%
%% The code below is generated by the tool at http://dl.acm.org/ccs.cfm.
%% Please copy and paste the code instead of the example below.
%%
\begin{CCSXML}
<ccs2012>
 <concept>
  <concept_id>10010520.10010553.10010562</concept_id>
  <concept_desc>Computer systems organization~Embedded systems</concept_desc>
  <concept_significance>500</concept_significance>
 </concept>
 <concept>
  <concept_id>10010520.10010575.10010755</concept_id>
  <concept_desc>Computer systems organization~Redundancy</concept_desc>
  <concept_significance>300</concept_significance>
 </concept>
 <concept>
  <concept_id>10010520.10010553.10010554</concept_id>
  <concept_desc>Computer systems organization~Robotics</concept_desc>
  <concept_significance>100</concept_significance>
 </concept>
 <concept>
  <concept_id>10003033.10003083.10003095</concept_id>
  <concept_desc>Networks~Network reliability</concept_desc>
  <concept_significance>100</concept_significance>
 </concept>
</ccs2012>
\end{CCSXML}

\ccsdesc[500]{Computer systems organization~Embedded systems}
\ccsdesc[300]{Computer systems organization~Redundancy}
\ccsdesc{Computer systems organization~Robotics}
\ccsdesc[100]{Networks~Network reliability}

%%
%% Keywords. The author(s) should pick words that accurately describe
%% the work being presented. Separate the keywords with commas.
\keywords{datasets, neural networks, gaze detection, text tagging}

%%
%% This command processes the author and affiliation and title
%% information and builds the first part of the formatted document.
\maketitle

\section{Introduction}

% Problem description and motivation
Most people who do outdoor activities run into the problem of finding an appropiate route. 
Depending on the activity from hiking and jogging to gravel and road cycling, requirements from users can greatly vary.
To this end we have developed \tM. 
The tool \tM constists out of an intuitive UI that allows users to create tours customed to their specific demands in their own webbrowser. Furthermore, \tM contains a few algorithms for computing solutions for the arc orienteering problem (AOP) and the more general touring problem.

For the touring problem there exist two distinct perspectives. The first is the perspective of maximizing profit as with the arc orienteering problem (AOP). 
Given a length budget and an profit function over every edge, find a route that does not exceed the budget and maximizes the total profit. 
An important detail to this approach is that the profit of an edge is only counted once, whilst the cost of an edge is added every time.
In an effort of finding better and nicer tours more attributes could be maximized. 
For example, we can maximize the total area covered by the tour, to get a tour that is closed to a circle as possible, or minimize turn cost as this requires slowing down.

The second perspective is perhaps more intuitive and involves minimizing cost. 
Here edges that are desirable are reduced in cost resulting in a new graph representing the underlying streets. 
A shortest path algorithm can be used to find a route between two points which will result in a route where desired edges are prioirtized by the algorithm as they are less costly to traverse.
In order to create a cyclic tour with the minimizing cost approach, intermediate waypoints need to be choosen. Between these waypoints a shortest path calculation is performed resulting in the 
 

\subsection{Related Work}
\subsection{Contribution}

\section{System}
The main contribution of this work is the framework that allows for easy 

\subsection{Architecture}

\subsection{Data}
To represent streets and roads as a graph we use the data from OpenStreetMap\footnote{https://www.openstreetmap.org}. The street data is thereafter processed by OSMnx \cite{}. The 

\subsubsection{Backbone}
In order to get exact solutions for routes of length larger than 2 kilometers we can choose to run these exact algorithms on a simplified graph.
We include roads in our backbone if they are part of a registered bike route in OSM. Registered bike routes are usually routes maintained by local and regional governments.
Therefore, for cycling, these usually give a reasonable network of roads and cycle ways that are fit to cycle on.
\subsection{Interface}

\section{Algorithm}

For this iteration we have implemented four dirrent algorithms for calulation and improvement of routes.

\subsection{Greedy Selection}
The greedy selection algorithm uses a straightforward greedy choice for selecting 

\subsection{Jogging Tour}\label{alg:jogging}

\subsection{Iterative Local Search}
We have implemented the iterative local search from Verbeeck et al. [?]. 
We however do not use the initialization phase of their algoirthm as this does not produce suffiently good results for us. 
Therefore, we run the Jogging tour alogirthm from Section \ref{alg:jogging} to obtain an intial solution. 
This solution is thereafter improved by the iterative local search using the time budget given by the user.

\subsection{Integer Linear Programming}

The integer linear program (ILP) gives the optimal solution for an instance of the AOP. The ILP used in \tM is a modified version from Verbeeck et al. 
The ILP from [] introduces a constraint for every subset of the vertices in order to avoind disconnected components, resulting in $\mathcal{O}(2^n)$ constrains.

\begin{equation}
  something Bad
  \label{exponential_ILP}
\end{equation}

The ILP from [] uses Equation \ref{exponential_ILP} to avoid subcycles. Instead we introduce a variable $\rho_{kij}$, for $1 \leq k \leq L$ and $1 \leq i, j \leq m$. Variable $\rho_{kij}$ denotes whether edge $e_{ij}$ is included in the path at location $k$.
\begin{align}
  \sum_{i=1}^m \sum_{j=1}^m \rho_{kij} = 1 &&\forall 1 \leq k \leq L \label{cons:at_most_one}\\
  \sum_{k=1}^L \rho_{kij} = \begin{cases} h_{ij} &\text{ if } e_{ij} \text{ is an edge} \\
    0 &\text{ otherwise}
  \end{cases} && \forall 1 \leq i, j \leq m \label{cons:sum_h_zero}\\
  2 \cdot \rho_{kij} \leq p[k][i] + p[k+1][j]
\end{align}
We include Constraint \ref{cons:at_most_one} for every $1 \leq k \leq L$ so that the path only has one edge at every position.

Constraint \ref{cons:sum_h_zero}.

\section{Conclusion}

% \begin{tabular}{rl}
%   $s$ & part of the input, start vertex\\
%   $t$ & part of the input, target vertex\\
%   $B$ & part of the input, the budget of the tour.\\
%   $c_{ij}$ & cost of traveling over edge $e_{ij}$.\\
%   $\pi_{ij}$ & profit of traveling over edge $e_{ij}$.\\\\
%   $b_{ij}$ & binary value for every edge $e_{ij}$, assume that $i < j$.\\
%   $n_{ij}$ & number of times that the edge $e_{ij}$ is included in the path.\\
%   $p_{ki}$ & true if $v_i$ is in position $k$.
% \end{tabular}

% \begin{equation*}
% \begin{array}{lr@{}ll}
% \text{maximize}  & \displaystyle\sum\limits_{e_{ij} \in E} \pi_{ij}b_{ij} & &\\\\
% \text{subject to}& \displaystyle\sum\limits_{e_{ij} \in E}   c_{ij}b_{ij} & \leq B  &\\
%                &                                         b_{ij}  & \leq  &\forall e_{ij} \in E\\
%                &                                          n_{ij} &\in \{0, \cdots, B\}, &\forall e_{ij} \in E
% \end{array}
% \end{equation*}

% \printbibliography

\end{document}
\endinput
%%
%% End of file `sample-acmsmall-biblatex.tex'.